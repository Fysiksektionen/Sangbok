\documentclass[a6paper,10pt]{article}
\usepackage[T1]{fontenc}
\usepackage[english,swedish]{babel}
\usepackage[utf8]{inputenc}
\usepackage{float, graphicx,amsmath,amsfonts,cite,enumerate,tabularx}
\usepackage[final]{pdfpages}
\usepackage{wrapfig}
\usepackage[margin=0.3in]{geometry}
\usepackage{sidspaltHack}
\newcommand{\mel}[1]{\small\textbf{\textit{mel. #1 \\}}}


\setlength{\oddsidemargin}{-0.37in}
\setlength{\textwidth}{225pt}

\pagestyle{empty}

\begin{document}
\nysida{0}{}
\begin{table}[!h]
\begin{tabularx}{1\textwidth}{l X}
4.&Minne\\
5.&Antisnapsvisa\\
6.&Dom som är nyktra\\
7.&Treo-comp\\
8.&Vit vecka\\
9.&Vi dricka, vi dricka\\
10.&When I get drunker\\
11.&Vi ska supa\\
&Härjarens bordsvisa\\
12.&Selen lever
\end{tabularx}
\end{table}
\vspace{-5pt}
\noindent
\Large $\iota$ (iota) - Torra visor
\vspace{-5pt}
\begin{table}[!h]
\begin{tabularx}{1\textwidth}{l X}
1.&Système International\\
2.&Integralkalylens fader\\
3.&GG-visan\\
4.&Öl sex\\
5.&The BASIC song\\
6.&Mors lilla dator\\
7.&Tentamenssång\\
8.&ODE till en husvagn\\
9.&Matlab\\
10.&Hållfvisa\\
11.&Elämnenas lov\\
12.&O hemska lab\\
&O hemska lab\\
&O hemska lab\\
13.&Aris summavisa\\
14.&Kvarkvisan\\
15.&Liten visa om Gram-Schmidts metod\\
\end{tabularx}
\end{table}
\begin{table}[!h]
\begin{tabularx}{1\textwidth}{l X}
16.&Kemisången\\
17.&Imperial system\\
18.&Jag gillar meken\\
19.&Termon\\
20.&Henelius-eufori\\
21.&Reglerteknik på bal\\
22.&En matematiker\\
23.&Det är långt bort till Alba Nova\\
24.&Tentapluggsblues\\
25.&Système Bolaget
\end{tabularx}
\end{table}

\newpage
\noindent
\Large $\kappa$ (kappa) - Fina visor
\vspace{-5pt}
\begin{table}[!h]
\begin{tabularx}{1\textwidth}{l X}
1.&Festen skall börjas\\
2.&Festvisa\\
3.&Sjösala vals\\
4.&Änglamark\\
5.&Fritiof och Carmencita\\
6.&Än en gång däran\\
7.&En liten blå förgätmigej\\
8.&Längtan till landet\\
9.&Balladen om Herr Fredrik Åkare och den söta fröken Cecilia
Lind\\
10.&Hårgalåten\\
11.&En dansk aquavit\\
12.&Tring, trink\\
13.&Smedsvisan\\
&Korta smedsvisan\\
14.&Balladen om ett kärlekspar\\
15.&Stad i ljus
\end{tabularx}
\end{table}
\newpage
\noindent
\Large $\lambda$ (lambda) - Vackra visor
\vspace{-5pt}
\begin{table}[!h]
\begin{tabularx}{1\textwidth}{l X}
1.&Fredmans sång n:o 21 - Måltidssång\\
2.&Kor ur Bacchi tempel\\
3.&Fredmans epistel n:o 48\\
4.&Molltoner från Norrland\\
5.&Den blomstertid\\
6.&Nu grönskar det \\
7.&Visa vid midsommartid\\
8.&Kristallen den fina\\
9.&Madrigal\\
10.&Glunt nr 25 - Examenssexa\\
11.&Sommarpsalm\\
12.&Uti vår hage\\
13.&Värmlandsvisan
\end{tabularx}
\end{table}

\vspace{-6pt}
\noindent
\Large $\mu$ (my) - Nidvisor
\vspace{-5pt}
\begin{table}[!h]
\begin{tabularx}{1\textwidth}{l X}
1.&Jag har aldrig vart på snusen\\
2.&Handelsvisa\\
3.&Fysikhatarvisan\\
$\pi$.&Matematikhatarvisan\\
4.&Datas visa\\
6.&Hyllningsvisa\\
7.&Man ska gå teknis \\
8.&Teknologvisa\\
9.&Teknologen och ekonomen
\end{tabularx}
\end{table}

\noindent
\Large $\nu$ (ny) - Skojiga visor
\vspace{-5pt}
\begin{table}[!h]
\begin{tabularx}{1\textwidth}{l X}
1.&Lingonben\\
\end{tabularx}
\end{table}
\begin{table}[!h]
\begin{tabularx}{1\textwidth}{l X}
2.&Älska dig själv\\
3.&Balladen om den kaxiga myran\\
4.&Nikolajev\\
5.&Danse Macabre\\
6.&Katten \\
7.&Undulaten \\
8.&Bakfyllosofen\\
9.&Mellansup\\
10.&Kanta Studjosi\\
11.&Indialand\\
12.&Zwampen\\
13.&Lumberjack song\\
14.&Under en filt i Madrid\\
15.&Hallen luta\\
16.&Styrman Karlssons äventyr med porslinspjäsen\\
17.&Sjung om Fru Svenssons lyckliga karl\\
18.&Jesus lever\\
37,5.&Temperaturen
\end{tabularx}
\end{table}

%\vspace{-5pt}
\noindent
\Large $o$ (omikron) - Visor till Fysiker
\vspace{-5pt}
\begin{table}[!h]
\begin{tabularx}{1\textwidth}{l X}
1.&Konglig Fysiks Paradhymn\\
2.&Årskursernas hederssång\\
3.&De Brevitate Vitae (Gaudeamus)\\
4.&Överföhssången\\
5.&Studentsången\\
6.&När Fumla blev fem\\
$\infty$.&O gamla klang och jubeltid
\end{tabularx}
\end{table}

\noindent
\Large $\sigma$ (sigma) - Visor vi minns

\newpage
\noindent
\Large $\omega$ (omega) - Noter
\vspace{-5pt}
\begin{table}[!h]
\begin{tabularx}{1\textwidth}{l X}
1.&Du gamla du fria\\
2.&Kungssången\\
3.&Sveriges flagga\\
4.&Porthos visa\\
5.&Lyft ditt välförsedda glas\\
6.&Längtan till landet\\
7.&Amanda Lundbom\\
8.&Smedsvisa\\
9.&Molltoner från Norrland\\
10.&Nu grönskar det\\
11.&Studentsången\\
$\infty$.&O gamla klang och jubeltid
\end{tabularx}
\end{table}


\noindent
\Large Register
\end{document}