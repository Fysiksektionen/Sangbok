\documentclass[a6paper,10pt]{article}
\usepackage[T1]{fontenc}
\usepackage[british]{babel}
\usepackage[utf8]{inputenc}
\usepackage{float, graphicx,amsmath,amsfonts,cite,enumerate,tabularx}
\usepackage[final]{pdfpages}
\usepackage[margin=0.3in]{geometry}

\setlength{\oddsidemargin}{-0.37in}
\setlength{\evensidemargin}{-0.47in}
\setlength{\textwidth}{215pt}

\pagestyle{empty}

\begin{document}
\noindent\Large Register
\vspace{5pt}\\
\small\textit{Kursiva rader är begynnelserader}
\begin{table}[!h]
\begin{tabular}{l l}
1, 2, 3, Whisky!&$\eta$ 12\\
37:an&$\beta$ 11\\
Angorakatten&$\delta$ 2c\\
Antisnapsvisa&$\theta$ 5\\
Aris summavisa&$\iota$ 13\\
Auld lang syne&$\alpha$ 5\\
Bakfyllosofen&$\nu$ 8\\
Balladen om den kaxiga myran&$\nu$ 3\\
Balladen om ett kärlekspar&$\kappa$ 14\\
Balladen om Herr Fredrik Åkare och&$\kappa$ 9 \\
den söta fröken Cecilia Lind&	\\
Bordeaux, Bordeaux&$\varepsilon$ 7\\
Brännvin är jävligt gott&$\delta$ 11d\\
Brännvin hit&$\delta$ 4b\\
Bär ner mig till sjön&$\theta$ 3\\
Danse Macabre&$\nu$ 5\\
Datas visa&$\mu$ 4\\
De Brevitate Vitae (Gaudeamus)&$o$ 3\\
Den blomstertid&$\lambda$ 5\\
Den vingklippta måsen&$\delta$ 10c\\
Det är långt bort till Alba Nova&$\iota$ 23\\
Djungelpunsch&$\zeta$ 7\\
Dom som är nyktra&$\theta$ 6\\
Du gamla du fria&$\alpha$ 1\\
Du gamla vin&$\varepsilon$ 5\\
Då verka lätt&$\delta$ 17b\\
Däj-o&$\eta$ 4\\
Elämnenas lov&$\iota$ 11\\
\end{tabular}
\end{table}

\newpage
\begin{table}[!h]
\begin{tabular}{l l}
Elysisk längtan&$\varepsilon$ 6\\
Emils spritvisa&$\beta$ 6\\
En cyklar för lite&	$\delta$ 5c\\
En dansk aquavit&$\kappa$ 11\\
En kan dricka vatten&	$\eta$ 2\\
En liten blå förgätmigej&$\kappa$ 7\\
En liten fyllhund&$\delta$ 11e\\
En matematiker&$\iota$ 22\\
En pilsnerdrickare&$\gamma$ 7\\
En ska gå Teknis&	$\mu$ 6\\
Et langue d'or&$\delta$ 1c\\
Feministvikingen&$\delta$ 16d\\
Festen skall börjas&$\kappa$ 1\\
Festvisa&$\kappa$ 2\\
Feta fransyskor&$\varepsilon$ 2\\
Finsk brännvinsvisa&$\delta$ 6b\\
Finsk snapsvisa&$\delta$ 6a\\
Fiskarna&$\delta$ 11b\\
Fkåne faft&$\delta$ 5a\\
Flaggpunschens visa&$\zeta$ 11\\
För att människan&$\delta$ 5d\\
Fredmans epistel n:o 48	&$\lambda$ 3\\
Fredmans sång n:o 21&$\lambda$ 1\\
Fritiof och Carmencita&$\kappa$ 5\\
Full är bäst&$\delta$ 7d\\
Full och galen&$\delta$ 7b\\
Fysikhatarvisan&$\mu$ 3\\
Gammalt brännvin&$\delta$ 15c\\
Gasqueljäsen&$\beta$ 9\\
Getingen&$\delta$ 11f\\
\end{tabular}
\end{table}

\newpage
\begin{table}[!h]
\begin{tabular}{l l}
GG-visan&$\iota$ 3\\
Glunt nr. 25&$\lambda$ 10\\
Gräv ur tundran&$\delta$ 8d\\
Gums visa&$\delta$ 4c\\
Göken&$\delta$ 15a\\
Hallen luta&$\nu$15\\
Halvan&$\delta$ 2b\\
Handelsvisa&$\mu$ 2\\
Hej på er vänner alla&$\beta$ 7\\
Helan går&$\delta$ 1a\\
Helan gick&$\delta$ 2d\\
Helan rasat&$\delta$ 2e\\
Hell and gore&$\delta$ 1b\\
Heltal rho&$\delta$ 1g\\
Henelius-eufori&$\iota$ 20\\
Humlorna&$\delta$ 11a\\
Hyllning till OP Andersson&$\delta$ 9a\\
Hyllningsvisa&$\mu$ 5\\
Hyllningsvisa till absinten&$\eta$ 7\\
Hållfvisa&$\iota$ 10\\
Hårgalåten&$\kappa$ 10\\
Häflåten&$\eta$ 9\\
Härjarens bordsvisa&$\theta$ 11b\\
Härjavisan&$\beta$ 3\\
Hörapparaten&$\delta$ 14e\\
Imbelupet&$\delta$ 1d\\
Imperial punsch&$\zeta$ 6\\
Imperial system&$\iota$ 17\\
Indialand&$\nu$ 11\\
Inre dialog&$\delta$ 9d\\
\end{tabular}
\end{table}

\newpage
\begin{table}[!h]
\begin{tabular}{l l}
Integralkalkylens fader&$\iota$ 2\\
Internationalen&$\alpha$ 4\\
Jag gillar meken&$\iota$ 18\\
Jag gillar punschen&$\zeta$ 5\\
Jag har aldrig vart på snusen&$\mu$ 1\\
Jag skall festa&$\beta$ 5\\
Jag var full en gång&$\theta$ 2\\
JASen&$\delta$ 10d\\
Jesus lever&$\nu$ 17\\
Ju mera öl vi dricker&$\gamma$ 10\\
Kaffe&$\eta$ 10\\
Kahlua&$\eta$ 13\\
Kalla små nubbar&$\delta$ 11g\\
Kalmarevisan&$\beta$ 4\\
Kanta Studjosi&$\nu$ 10\\
Katten&$\nu$ 6\\
Kemisången&$\iota$ 16\\
Konglig Fysiks Paradhymn&$o$ 1\\
Kor ur Bacchi tempel	&$\lambda$ 2\\
Korta smedsvisan&	$\kappa$ 13b\\
Korta solen	&$\delta$ 14b\\
Kristallen den fina&	$\lambda$ 8\\
Krök armen	&$\delta$ 9c\\
Kungssången&	$\alpha$ 2\\
Kvarkvisan	&$\iota$ 14\\
Lapin Kulta&$\gamma$ 9\\
Lille Olle	&$\beta$ 8\\
Lingonben	&$\nu$ 1\\
Liten visa om Gram-Schmidts metod	&$\iota$ 15\\
Livet är härligt&	$\delta$ 8a\\
\end{tabular}
\end{table}

\newpage
\begin{table}[!h]
\begin{tabular}{l l}
Lumberjack song	&$\nu$ 13\\
Lundströms kök	&$\delta$ 13e\\
Lyft ditt välförsedda glas&	$\beta$ 2\\
Längtan till landet&	$\kappa$ 8\\
Madrigal&	$\lambda$ 9\\
Magen brummar&	$\delta$ 15b\\
Matematikhatarvisan& $\mu$ $\pi$\\
Matlab	&$\iota$ 9\\
Mattevisan&	$\beta$ 10\\
Mellansup	&$\nu$ 9\\
Mera järn&	$\delta$ 12c\\
Mera Skåne	&$\delta$ 5b\\
Mesen&	$\delta$ 10h\\
Min pilsner	&$\gamma$ 4\\
Minne&	$\theta$ 4\\
Mjölk	&$\eta$ 5\\
Mjölksång&	$\eta$ 6\\
Mod i barm&$\delta$ 12a\\
Molltoner från Norrland&	$\lambda$ 4\\
Moose:en	&$\delta$ 10f\\
Mors lilla dator&	$\iota$ 6\\
Morsgrisar små&	$\delta$ 7e\\
Musen	&$\delta$ 10i\\
Månen (En gång i månan)&	$\delta$ 15d\\
Månvisa&	$\delta$ 10a\\
Måsen	&$\delta$ 10b\\
Måsens sista sup&	$\delta$ 10j\\
Nikolajev	&$\nu$ 4\\
Nu grönskar det&	$\lambda$ 6\\
Nu ska vi klämma septen	&$\delta$ 7a\\
Nu tar vi rom&	$\eta$ 3\\
\end{tabular}
\end{table}

\newpage
\begin{table}[!h]
\begin{tabular}{l l}
Nubbekantat&	$\delta$ 3a\\
När Fumla blev fem&  $o$ 6\\
När helan en tagit	&$\delta$ 2a\\
När jag är fuller&	$\delta$ 10g\\
När nubben blänker&	$\delta$ 10e\\
O gamla klang och jubeltid&	$o$ $\infty$\\
O hemska lab&	$\iota$ 12a\\
O hemska lab&	$\iota$ 12b\\
O hemska lab&	$\iota$ 12c\\
ODE till en husvagn	&$\iota$ 8\\
Ode till ölet&	$\gamma$ 3\\
Oh Besinna&	$\delta$ 12b\\
Old Janx Spirit&	$\delta$ 14c\\
Planksaft	&$\delta$ 4a\\
Porthos visa&	$\beta$ 1\\
Portvins visa	&$\varepsilon$ 10\\
Punsch, punsch	&$\zeta$ 9\\
Punschen kommer&	$\zeta$ 1\\
Punschens lov&	$\zeta$ 4\\
Punschfinalen	&$\zeta$ 1/$\varepsilon$\\
Punschkanon&	$\zeta$ 2\\
Punschschottis	&$\zeta$ 3\\
Raj-Raj&	$\delta$ 16a\\
Reglerteknik på bal $\iota$ 21\\
Räven&	$\delta$ 6d\\
Röd vitamin	&$\varepsilon$ 9\\
Röda vinet&	$\varepsilon$ 11a\\
Sanningen om ölet&	$\gamma$ 5\\
Schottis på Valhall&	$\eta$ 8\\
Selen lever	&$\theta$ 12\\
\end{tabular}
\end{table}

\newpage
\begin{table}[!h]
\begin{tabular}{l l}
Sista punschvisan	&$\zeta$ $\infty$\\
Sjösala vals	&$\kappa$ 3\\
Sjung om Fru Svenssons lyckliga karl&	$\nu$ 17\\
Skål för vatnnet&	$\eta$ 1\\
Små nubbarna&	$\delta$ 3c\\
Smedsvisan	&$\kappa$ 13\\
Solen	&$\delta$ 14a\\
Sommarpsalm	&$\lambda$ 11\\
Spegelvisa&	$\varepsilon$ 8\\
Stad i ljus&       $\kappa$ 15\\
Störthärligt full&	$\theta$ 1\\
Strejk på Pripps&	$\gamma$ 2\\
Studentsången&	$o$ 5\\
Studiemedelsrondo&$\zeta$ 10\\
Styrman Karlssons äventyr &$\nu$ 16\\
med porslinspjäsen	&\\
Supen	&$\delta$ 6e\\
Sveriges Arraktionalhymn	&$\zeta$ 13\\
Sveriges flagga	&$\alpha$ 3\\
Système Bolaget&	$\iota$ 25\\
Système International&	$\iota$ 1\\
Så hastigt	&$\delta$ 8c\\
Sädesfälten	&$\delta$ 6c\\
Sänkta Lucia	&$\delta$ 13d\\
Teknologen och ekonomen&$\mu$ 8\\
Teknologvisa	&$\mu$ 7\\
Temperaturen	&$\nu$ 37,5\\
Tentamenssång&	$\iota$ 7\\
Tentapluggsblues&$\iota$ 24\\
\end{tabular}
\end{table}

\newpage
\begin{table}[!h]
\begin{tabular}{l l}
Termon&	$\iota$ 19\\
The BASIC song	&$\iota$ 5\\
Till supen så tager en sill	&$\delta$ 13b\\
Tjugotre	&$\delta$ 16b\\
Toj hemtegubbar	&$\delta$ 7c\\
Treo-comp	&$\theta$ 7\\
Trink, Trink&	$\kappa$ 12\\
Tänk om jag hade lilla nubben&	$\delta$ 9b\\
Ubåten	&$\delta$ 1f\\
Under en filt i Madrid	&$\nu$ 14\\
Undulaten&	$\nu$ 7\\
Uti min mage	&$\delta$ 17a\\
Uti vår hage	&$\lambda$ 12\\
Var Osquristina	&$\delta$ 3b\\
Var redo!	&$\delta$ 13a\\
Vem kan hugga	&$\delta$ 12e\\
Vem kan kröka	&$\delta$ 12d\\
Vem sade ordet skål?	&$\delta$ 1e\\
Vi dricka, vi dricka&	$\theta$ 10\\
Vi går över ån	&$\delta$ 13c\\
Vi ska supa	&$\theta$ 12\\
Vi vill ha punsch	&$\zeta$ 8\\
Vi älskar öl	&$\gamma$ 11\\
Vikingen	&$\delta$ 16c\\
Vinet skänker	&$\varepsilon$ 4\\
Vinet väntar	&$\varepsilon$ 11b\\
Vinets lov	&$\varepsilon$ 1\\
Vinvisa	&$\varepsilon$ 3\\
Visa en torsdagskväll	&$\zeta$ 12\\
Visa vid midsommartid &$\lambda$ 7\\
\end{tabular}
\end{table}
\newpage
\begin{table}[!h]
\begin{tabular}{l l}
Vit vecka	&$\theta$ 8\\
Vodka, vodka	&$\delta$ 8b\\
Värmlandsvisan	&$\lambda$ 13\\
When I get drunker	&$\theta$ 10\\
Whiskyn	&$\eta$ 11\\
Zwampen&	$\nu$ 12\\
Årskursernas hederssång&$o$ 2\\
Älska dig själv&$\nu$ 2\\
Än en gång däran&$\kappa$ 6\\
Änglamark&$\kappa$ 4\\
Änglarna&$\delta$ 11c\\
Öl sex&	$\iota$ 4\\
Öl, öl, öl i glas	&$\gamma$ 8\\
Ölbytarvisan	&$\gamma$ 1\\
Ölvisan&$\gamma$ 6\\
Överföhssång&	$o$ 4\\
\vspace{180pt}
\end{tabular}
\end{table}
\end{document}