\documentclass[a6paper, 10pt, twoside]{article}
%\usepackage[T1]{fontenc}
\usepackage[british]{babel}
\usepackage[utf8]{inputenc}
\usepackage{float, graphicx,amsmath,amsfonts,cite,enumerate,tabularx}
\usepackage[final]{pdfpages}
\usepackage{wrapfig}
\usepackage[margin=0.3in]{geometry}
\usepackage{sidspaltHack}
\usepackage{digital}
\usepackage{ulem}


\setlength{\evensidemargin}{-0.47in}
\setlength{\oddsidemargin}{-0.37in}
\setlength{\textwidth}{215pt}

\pagestyle{empty}

\begin{document}
\nysida{15}{}
\noindent
\chaptertitlenobr{$\Sigma\sigma$}{Visor vi minns}
\small
\vspace{10pt}

\noindent
\small
Du har nått det sista sångkapitlet: Sigma! Målet med detta kapitel är att ge en bild av det gångna året på Fysiksektionen, eller snarare en ljudbild av gasquer, banquetter, spex eller andra tillställningar.  

Häri finner du ett axplock av sånger som lyckats fångas upp. Det är inte i närheten av alla fantastiska gyckel och nya sånger som dykt upp under läsåret 2023/2024 som finns med här, men förhoppningsvis kan det ändå delvis återspegla året. 

Vill du få med en sång till nästa års sångbok? Tveka inte då att tipsa nästa års sångboksansvarig och framför din åsikt! Många av texterna du finner här är med tack vare ivriga tips och hjälp från andra fysiker och matematiker. Se till att sjunga på ordentligt under året, så att nästa års $\sigma$-kapitel blir minst lika spektakulärt!

Jag önskar er mycket sångglädje och all annan form av glädje!
\auth{Christoffer Ousbäck \\Sångboksansvarig 2024}

\nysida{15}{24.1}

\begin{center}
\huge{\textit{2023/2024}}
\end{center}

\begin{center}
    \songtitle{$\sigma24.1$}{Fattig student}
    \mel{Fattig bonddräng}
    \textit{Till minne av Georg Riedel}
\end{center}
\begin{lyrics}
Jag är fattig student \\
men jag lever ändå. \\
Dagar går och kommer \\
medan jag pluggar på, \\
räknar, programmerar, \\
stressar, skriker och svär. \\
Snart så kommer tentan, \\
vilken jävla misär! \\
\vspace{2pt}\\
För att dämpa oron \\
tar jag gärna en sup, \\
nu när helgen kommit \\
då vill jag ha det kul. \\
Plugga hela helgen? \\ 
Det orkar man ej.\\
Jag ska ut och festa, \\
och dansa med dig!\\
\vspace{2pt}\\
Men så kommer måndan, \\
varför var jag på fest? \\
Tentan börjar åtta, \\
det får bli en till rest. \\
Ändå ska det firas, \\ 
nu är tentorna slut! \\
Nu kan fattig student, \\
äntligen vila ut.

\end{lyrics}
\auth{Erik Westergren}

\nysida{15}{24.2}

\begin{center}
    \songtitle{$\sigma24.2$}{Å se funktionen då}
    \mel{The Rattlin' Bog} 
\end{center}
\begin{lyrics}
$\|$: Å se funktionen då \\
Vi integrerar funktionen då :$\|$ \\
I funktionen finns ett bråk, \\
ett fint bråk, ett sällsynt bråk\\ 
Ett bråk i en funktion \\
och vi integrerar funktionen då \\
\vspace{7pt}\\
$\|$: Å se... :$\|$ \\
I det bråket finns en täljare, \\
en fin täljare en sällsynt täljare\\
Ett täljare i ett bråk, ett bråk i en funktion \\
och vi integrerar funktionen då \\
\vspace{7pt}\\
$\|$: Å se... :$\|$ \\
I den täljarn finns en cos, \\ 
en fin cos en sällsynt cos \\
En cos i en täljare, en täljare i ett bråk, \\
ett bråk i en funktion \\
och vi integrerar funktionen då \\
\vspace{7pt}\\
$\|$: Å se... :$\|$ \\
I den cos så finns ett theta, \\
ett fint theta, ett sällsynt theta. \\
Ett theta i en cos, en cos i en täljare, \\
en täljare i ett bråk, ett bråk i en funktion \\
och vi integrerar funktionen då\\
\vspace{7pt}\\
$\|$: Å se... :$\|$ \\
I vårt theta finns polynom, \\
ett femte gradens polynom \\
Ett polynom i ett theta, ett theta i en cos, \\
en cos i en täljare, en täljare i ett bråk, \\
ett bråk i en funktion \\
och vi integrerar funktionen då\\
\vspace{7pt}\\
$\|$: Å se... :$\|$ \\
I polynom finns en rot, \\
en ful rot, en sjunde rot. \\
En rot i ett polynom, ett polynom i ett theta, \\
ett theta i en cos, en cos i en täljare, \\
en täljare i ett bråk, ett bråk i en funktion \\
och vi integrerar funktionen då \\
\vspace{7pt}\\
$\|$: Å se... :$\|$ \\
Men i roten finns en kub, \\
en dum kub, en jävlig kub! \\
En kub i en rot, en rot i ett polynom, \\
ett polynom i ett theta, ett theta i en cos, \\
en cos i en täljare, en täljare i ett bråk, \\
ett bråk i en funktion \\
och vi integrerar funktionen ändå \\
\vspace{7pt}\\
$\|$: Å se... :$\|$ \\
Men i kuben finns ett $i$, \\
f*cking $i$, \textbf{VARFÖR $i$}. \\
Ett $i$ i en kub, en kub i en rot, \\
en rot i ett polynom, ett polynom i ett theta, \\
ett theta i en cos, en cos i en täljare, \\
en täljare i ett bråk, ett bråk i en funktion \\
och vi integrerar funktionen ändå \\
\vspace{7pt}\\
$\|$: Å se... :$\|$ \\
Bredvid $i$:t finns en funktion, \\
en \textit{TILL?} \\
\vspace{-2pt}\\
...\\
\vspace{1pt}\\
En funktion vid ett $i$, ett $i$ i en kub, \\
en kub i en rot, en rot i ett polynom, \\
ett polynom i ett theta, ett theta i en cos, \\
en cos i en täljare, en täljare i ett bråk, \\
ett bråk i en funktion \\
och jag kan icke integrera då! \\
\vspace{2pt}\\
Å se funktionen då, \\
jag lämnar den åt Wolfram då. \\
Å se funktionen då, \\
jag tar tentan till våren då! 

\end{lyrics}
\auth{Albin, Adrian, Lowe och Nessim,  F-22}

\nysida{15}{24.3}

\begin{center}
    \songtitle{$\sigma24.3$}{Fomorerna}
    \mel{Små grodorna}
\end{center}
\begin{lyrics}
Fomorerna, \\
fomorerna \\
är lustiga att se. \\
Fomorerna, \\
fomorerna \\
är lustiga att se.\\
\vspace{1pt}\\
Ej hjärna, ej hjärna, \\
men svansar hava de. \\
Ej hjärna, ej hjärna, \\
men lansar hava de. \\
\vspace{1pt}\\
Å hack, hack, hack \\
Å hack, hack, hack \\
Å hack, hack, hack, hack, aa \\
Å hack, hack, hack \\
Å hack, hack, hack \\
Å hack, hack, hack, hack, aa
\end{lyrics}
\auth{Fysikalen S:t Patrik 2023}

\begin{figure}[!h]
\hspace{150pt}
\includegraphics[width=0.2\textwidth]{fysikalen2023.png}
\end{figure}

\end{document}