\documentclass[a6paper, 10pt, twoside]{article}
%\usepackage[T1]{fontenc}
\usepackage[british]{babel}
\usepackage[utf8]{inputenc}
\usepackage{float, graphicx,amsmath,amsfonts,cite,enumerate,tabularx}
\usepackage[final]{pdfpages}
\usepackage{wrapfig}
\usepackage[margin=0.3in]{geometry}
\usepackage{sidspaltHack}
\usepackage{digital}
\usepackage{ulem}


\setlength{\evensidemargin}{-0.47in}
\setlength{\oddsidemargin}{-0.37in}
\setlength{\textwidth}{215pt}

\pagestyle{empty}

\begin{document}
\nysida{15}{}
\noindent
\chaptertitlenobr{$\Sigma\sigma$}{Visor vi minns}
\small
\vspace{10pt}

\noindent
\small
Du har nått det sista sångkapitlet. Sigma! 

Häri finner du ett axplock av visor från den gångna året som försöker
{\it summera} de upplevelser vi haft den senaste tiden på Fysiksektionen.
Det är inte i närheten av alla fantastiska gyckel och nya sånger som dykt
upp under läsåret 2022/2023 som finns med här, men en fingervisning om
det som lämnat F-teknologernas strupar hittar du på kommande sidor. 

Vill du få med ett gyckel/en sångtext/en fin visa till nästa års summering?
Tveka inte då att tipsa nästa års sångboksansvarig och framför din åsikt!
Många av texterna du finner här är med tack vare ivriga tips och hjälp från andra fysiker.

Jag önskar er mycket sångglädje och nöje i resan genom historien!
\auth{Valdemar Melin \\Fcoms ordförande 2023}

\nysida{15}{23.1}

\begin{center}
\huge{\textit{2022/2023}}
\end{center}

\begin{center}
    \songtitle{$\sigma23.1$}{Hej Toast}
    \mel{Hey Jude --- The Beatles} 
\end{center}
\begin{lyrics}
Hej Toast, vi älskar dig\\
Även om vi gör er helt insane\\
Kom ihåg att bli inte allt för sur\\
För vi anser det tur att ha er som värdar\\
\vspace{5pt}\\
Hej Toast, bli inte arg\\
När inget gått som planerat\\
Det är den läxa vi hoppas vi lärt\\
förlåt om vi tärt en bit av er hälsa\\
\vspace{5pt}\\
Och känner ni att allt står still, hej Toast, säg till\\
Så ska vi försöka att lugna ner oss\\
syftet är ej att ni tar på allt ansvar\\
För hur vi alla på gasquen beter oss\\
Na-na-na-na-na, na-na-na-na\\
\vspace{5pt}\\
Hej Toast, försök va stark\\
vissa stunder kan allt se hejdlöst\\
Men i slutet av tunneln finns det hopp\\
Och för er tropp är vår kärlek ändlöst\\
\vspace{5pt}\\
Så känner ni att allt går trögt, hej Toast, skrik högt\\
Så kommer ert budskap fram till alla\\
Och vet ni ej att det är ni, hej Toast, ta i\\
Vi är er flock så börja valla\\
Na-na-na-na-na, na-na-na-na\\
\vspace{5pt}\\
Hej Toast, vi älskar dig\\
Även om vi gör er helt insane\\
Kom ihåg att bli inte allt för sur\\
För vi anser det tur att ha er som\\
Värdar, värdar, värdar, värdar, värdar, värdar!\\
\vspace{5pt}\\
$\|$: Na-na-na-na-na-na-na, na-na-na-na, hej Toast :$\|$ (17 ggr, eller tills vi bestämmer för att fadea ut)
\end{lyrics}
\auth{Benjamin Velin, F-21\\Anarkigasquen 2022}

\nysida{15}{23.2}

\begin{center}
    \songtitle{$\sigma23.2$}{Här på Konsulatet}
    \mel{Hotel California --- Eagles} 
\end{center}
\begin{lyrics}
Går längs Valhallavägen\\
Är så frusen och tärd\\
När jag närmar mig slutet\\
Av min pilgrimsfärd\\
\vspace{2pt}\\
Tar en avstickargata\\
Förbi östra station\\
Där står människor och svamlar om\\
Schrödingers ekvation\\
\vspace{2pt}\\
Sen passerar jag nymble\\
Det står en dörr på glänt\\
Där nånstans sliter THS\\
Representanter för varje student\\
\vspace{2pt}\\
Går en liten bit längre\\
Tills jag ser förgätmigej\\
Plötsligt står där en orangeklädd man\\
Och han hälsar mig\\
\vspace{2pt}\\
Välkommen \\
Hit till konsultatet\\
En sån vacker plats\\
Vår sektions palats\\
\vspace{2pt}\\
Pub varje tisdagkväll\\
Här på konsulatet\\
Både höst och vår \\
Nästan varje år\\
\vspace{2pt}\\
Den orangeklädda mannen\\
Har en tunga av brons\\
Övertalar mig att söka fysik\\
Så jag också kan hänga i kons\\
\vspace{2pt}\\
Här finner du svaren\\
På de frågor du har\\
I gröna skenet från en fysikers drink\\
I fkm*s fina bar\\
\vspace{2pt}\\
Så jag flyttar till Stockholm\\
En bit utanför stan\\
Hyr jag en etta på 24 kvadrat\\
För 10 000 om dan’\\
\vspace{2pt}\\
Jag har knappt någon inkomst\\
Varje krona går åt\\
Jag jobbar gratis på varendaste gasque\\
Bara för att ha råd\\
\vspace{2pt}\\
Välkommen \\
Hit till konsulatet\\
Ingen metanol\\
I vår alkohol\\
\vspace{5pt}\\
Coola trafikljus\\
Här på konsulatet\\
Varför stå i kö\\
När man kan va’ slö?\\
\vspace{2pt}\\
Vi må vara skumma\\
Men vi är ingen sekt\\
Bara för att vi har gratisarbete\\
Eget slang och en heltäcknings-dräkt\\
\vspace{2pt}\\
Det är helt demokratiskt\\
Här har alla en röst\\
Men när allting klingar unisont\\
Så är det föga tröst\\
\vspace{2pt}\\
Mina gamla vänner\\
Är det år sen jag såg\\
Det är sällan jag pratar med nån\\
Som inte är F-teknolog\\
\vspace{2pt}\\
Har du gått över gränsen\\
Blir du aldrig dig lik\\
Du kan exa eller hoppa av\\
Men du är alltid fysik\\
\vspace{2pt}\\
Välkommen\\
Hit till konsulatet\\
Vi finns här för dig\\
När det skiter sig\\
\vspace{2pt}\\
F:ar du tentan\\
Så kom till konsulatet\\
Det går nästa gång\\
Nu till nästa sång\\
\end{lyrics}
\auth{Robert Olsson Kihlborg F-21 \\ Jubelbanquetten 2022}

\nysida{15}{23.3}

\begin{center}
    \songtitle{$\sigma23.3$}{Golvcheck}
    \mel{Golvcheck} 
\end{center}
\begin{lyrics}
(Golvcheck)\\
(Golvet)\\
(Golvcheck)\\
\vspace{5pt}\\
Yuh\\
Ja hej, eh har du golvet med dig där?\\
En sekund\\
\vspace{5pt}\\
Golvcheck golvcheck, kolla ner på golvet\\
Är det under fötterna, vet du om det håller?\\
Golvcheck golvcheck, kolla ner på golvet\\
Är det under fötterna? (vet du om det håller?)\\
\vspace{5pt}\\
Det finns två små strån som är intressant\\
Får din tå stå på nått som liknar mark\\
Är det under din fot och så framförallt\\
Kan du banka dig blå till vår dundertakt\\
Och kan din rumpa förstå att det gungar tight\\
Så vi bankar på bås och vi slungar vibes\\
Vi är kungar man, vi ba rullar fram\\
När det är dags för en check, fucking checka allt\\
\vspace{5pt}\\
Har funnits sen urminnes tider\\
TØNDØV vi sprider\\
Vi skriver sidor\\
Håller oss vid livet\\
Ger dig kunskapen, undviker konstapeln\\
Visa mig kapen så lappar jag taken\\
\vspace{5pt}\\
Ta en skylt ta en till, många fler ska bli min\\
Ta en skylt ta en till, många fler ska bli min\\
Ta en skylt ta en till, många fler ska bli min\\
Nordkoreas ambassad, vet du var den finns?\\
\vspace{5pt}\\
Golvcheck golvcheck, kolla ner på golvet\\
Är det under fötterna, vet du om det håller?\\
Golvcheck golvcheck, kolla ner på golvet\\
Är det under fötterna, vet du om det håller?\\
Golvcheck golvcheck, kolla ner på golvet\\
Är det under fötterna, vet du om det håller?\\
Golvcheck golvcheck, kolla ner på golvet\\
Är det under fötterna? (vet du om det håller?)\\
\vspace{5pt}\\
Vi kommer ej tillbaka, vi har aldrig varit här\\
De säger checken saknar mening, man bör alltid vara säk\\
Lägger golvcheck upp i taket bara checkar var det är\\
Mannen TØNDØV sätter tonen ut till ljudet av gevär\\
\vspace{5pt}\\
Vi baxa en skylt, den var galet vacker\\
Satt i rondellen, fram tills på natten\\
Jag och mina mannar\\
Druckit litegranna\\
När vi såg en skylt, fick idén, börja sammla\\
\vspace{5pt}\\
Ta en skylt ta en till, många fler ska bli min\\
Ta en skylt ta en till, många fler ska bli min\\
Ta en skylt ta en till, många fler ska bli min\\
Nordkoreas ambassad, vet du var den finns?\\
\vspace{5pt}\\
(Golvcheck golvcheck, kolla ner på golvet)\\
(Är det under fötterna, vet du om det håller?)\\
Golvcheck golvcheck, kolla ner på golvet\\
Är det under fötterna, vet du om det håller?\\
Golvcheck golvcheck, kolla ner på golvet\\
Är det under fötterna, vet du om det håller?\\
Golvcheck golvcheck, kolla ner på golvet\\
Är det under fötterna, vet du om det håller?\\
Golvcheck golvcheck, kolla ner på golvet\\
Är det under fötterna? (vet du om det håller?)\\
\vspace{5pt}\\
GOLVCHECK…\\
GOLV\\
\vspace{5pt}\\
(Vet du om det håller?)\\
(Golvcheck)\\
(Vet du om det håller?)\\
(Golvcheck)\\
\vspace{5pt}\\
Japp, golvet är där\\
\end{lyrics}
\auth{Martin Siklosi, F-21}

\nysida{15}{23.4}

\begin{center}
    \songtitle{$\sigma23.4$}{FSN}
    \mel{Basket Case --- Green Day} 
\end{center}
\begin{lyrics}
$[$Vers 1$]$\\
Har du en sekund\\
Att ta en liten stund\\
Och lyssna på mitt klagomål om tentan\\
\vspace{5pt}\\
Jag är inte nöjd\\
Med rättningen jag fick\\
Jag borde fått ett D men jag fick F\\
\vspace{5pt}\\
$[$Ref 1$]$\\
Ja första frågan den var svår\\
För man fick tiden mätt i år\\
Ja alltför höga krav\\
För tiden som vi gavs\\
\vspace{5pt}\\
Det här ska jag ta upp\\
Med FSN\\
\vspace{5pt}\\
$[$Vers 2$]$\\
Men månaderna går\\
En omtenta blir två\\
Varenda gång jag skriver den går nåt snett\\
\vspace{5pt}\\
Jag börjar tappa tron\\
Har ingen ambition\\
Om att nånsin bli utexaminerad\\
\vspace{5pt}\\
$[$Ref 2$]$\\
Den fjärde omtentan känns bra\\
Jag borde verkligen få A\\
Men ladok säger nej\\
Att godkänd är jag ej!\\
Vart vänder man sig nu?\\
Ja svaret vet du nu! \\
Det här är ett problem\\
För FSN\\
\end{lyrics}
\auth{Robert Olsson Kihlborg, F-21 \\Välkomstgasquen 2022}


\nysida{15}{23.5}

\begin{center}
    \songtitle{$\sigma23.5$}{Sov Läraren}
    \mel{Sov lilla Totte --- Hasse \& Tage} 
\end{center}
\begin{lyrics}
Sov nu magistern\\
Nu varenda mister\\
Slumrar för natten den skön\\
Här får lärarn lön\\
\vspace{5pt}\\
Om ögon du sluter\\
Drömmer du och njuter\\
Av all skäll till elever gett\\
fått från skolan avgett\\
\vspace{5pt}\\
Tänk så roligt du har haft idag.\\
Tänk på alla stackars elever som du skrämt,\\
tänk på alla bevis på tentan kommer du sa på skämt.\\
Dröm om alla dem du nekat CSN\\
Dröm alla bostäder du sen köpt från marknaden\\
Visst var dem dyra, men du går plus när du hyr ut dem sen.\\
\vspace{5pt}\\
Sov nu lärarn’\\
Nu varenda elev i världen\\
I huvudet mardrömmar nu se\\
Om en ond lärare\\
\vspace{5pt}\\
I drömmens rike\\
Du tillsammans med fysik e´\\
Tyranner över stress och plugg\\
Elever får inte lugn ett dugg\\
\vspace{5pt}\\
I dina drömmar\\
Går elever runt och ömmar\\
Gråtandes studenter må du tro\\
Vyssar dig till ro.\\
\vspace{5pt}\\
Allt i drömmen ter sig rosenrött\\
Alla lärlingar har du skrämt så dom har dött.\\
tänk vad läraren jobbar, nu är lilla läraren trött.\\
Dröm om alla läxor du har gett ut\\
Dröm om alla elever du gjort så fritid de inte få\\
ja i dina drömmar är din himmel blå.\\
\vspace{5pt}\\
Sov nu lärarn’\\
Nu varenda elev i världen\\
I huvudet mardrömmar nu se\\
Om en ond lärare
\end{lyrics}
\auth{Benjamin Velin, F-21\\Gasque på Sängen 2023}

\end{document}
