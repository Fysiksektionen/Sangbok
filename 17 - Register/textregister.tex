\documentclass[a6paper,10pt]{article}
%\usepackage[T1]{fontenc}
\usepackage[british]{babel}
\usepackage[utf8]{inputenc}
\usepackage{float, graphicx,amsmath,amsfonts,cite,enumerate,tabularx}
\usepackage[final]{pdfpages}
\usepackage[margin=0.3in]{geometry}

\setlength{\oddsidemargin}{-0.37in}
\setlength{\textwidth}{225pt}

\pagestyle{empty}

\begin{document}
\noindent\Large Register
\begin{table}[!h]
\begin{tabular}{l l}
\textit{10 LET oss nu fatta i våra glas}	&$\iota$ 5\\
\textit{Ack, Värmeland, du sköna du härliga land!}	&$\lambda$ 13\\
\textit{Aftonrodnan svalka sprider}	&$\varepsilon$ 6\\
\textit{Alla de som går på Data}	&$\mu$ 4\\
\textit{sitter mest och kodar Perl-script varje natt}&\\
\textit{Att dricka brännvin är en sed}	&$\delta$ 6b\\
\textit{Att höja en skål kan nog vara rätt}	&$\delta$ 3a\\
\textit{Bär ner mig till sjön}	&$\theta$ 3\\
\textit{Bluff och Spark och Tork och Kvark}	&$\nu$ 1\\
\textit{Bort allt vad oro gör,}	&$\lambda$ 2\\
\textit{bort allt vad hjärtat kväljer}&\\
\textit{Brännvin hit och brännvin dit}	&$\delta$ 4b\\
\textit{Brännvin, öl, och gammal finkel}	&$\delta$ 14e\\
\textit{Däj-o, däääj-o, daylight come...}	&$\eta$ 4\\
\textit{Den blomstertid nu kommer}	&$\lambda$ 5\\
\textit{med lust och fägring stor}&\\
\textit{Den mås som satt på en klyvarbom}	&$\delta$ 10j\\
\textit{Det är dags för mellansup}	&$\nu$ 9\\
\textit{om det så är det sista jag gör}&\\
\textit{Det är torsdag morgon,}&$\iota$ 24\\
\textit{och mitt huvud känns så tungt}&\\
\textit{Det flög en JAS över Västerbron}	&$\delta$ 10d\\
\textit{Det satt en älg i en klyvartopp}	&$\delta$ 10f\\
\textit{Det satt en mås på klyvarbom}	&$\delta$ 10b\\
\textit{Det satt en mås på klyvarbom}	&$\delta$ 10c\\
\textit{Det satt en mes i en klyvarmast}	&$\delta$ 10h\\
\textit{Det satt en mus i en hushållsost}	&$\delta$ 10i\\
\textit{Det var en gång en vanlig bonnakatt, fallera}	&$\delta$ 2c\\
\textit{Det verkade lätt, att gå på KTH}     &o 6\\
\end{tabular}
\end{table}

\newpage
\setlength{\oddsidemargin}{-0.67in}
\begin{table}[!h]
\begin{tabular}{l l}
\textit{Dom som är nyktra de har inge' roligt}	&$\theta$ 6\\
\textit{Dragspel, fiol och mandolin}	&$\kappa$ 5\\
\textit{Dricka pilsner varje da, de e kul å de e bra}	&$\gamma$ 6\\
\textit{Du ädla, du friska, du livselixir}	&$\zeta$ 13\\
\textit{Du gamla du fria, du fjällhöga nord}	&$\alpha$ 1\\
\textit{Du gamla, du fina, du årgågna vin}	&$\varepsilon$ 5\\
\textit{Du häller ur flaskan en gyllene tår}	&$\zeta$ 12\\
\textit{Du lindar av olvon en midsommarkrans}	&$\lambda$ 7\\
\textit{och hänger den om ditt hår} &\\
\textit{Du ska inte tro att en summa}	&$\iota$ 13\\
\textit{blir alls vad den ser ut att va'} &\\
\textit{En cyklar för lite, en röker för mycke'}	&$\delta$ 5c\\
\textit{En gång i månan är månen full}	&$\delta$ 15d\\
\textit{En gång i min ungdom älskade jag,}	&$\kappa$ 13b\\
\textit{sa tack och adjö och försvann} &\\
\textit{En gång i min ungdom älskade jag}	&$\kappa$ 13a\\
\textit{en flicka med rena och sköna behag} &\\
\textit{En kan dricka vatten,}	&$\eta$ 2\\
\textit{mjölk, och gammalt flott} &\\
\textit{En liten enkel integral}	&$\iota$ 4\\
\textit{uti ett Vektor III-tal} &\\
\textit{En liten fyllhund på krogen satt}	&$\delta$ 11e\\
\textit{En matematiker här bor i staden} &$\iota$ 22\\
\textit{En pilsnerdrickare här bor i staden}	&$\gamma$ 7\\
\textit{En vänlig grönskas rika dräkt}	&$\lambda$ 11\\
\textit{har smyckat dal och ängar} &\\
\textit{En viking viker tvätten själv, hurra} &$\delta$ 16d\\
\textit{En viking älskar livets vand, hurra}	&$\delta$ 16c\\
\textit{Eskimåer jagar valross.}	&$\nu$ 8\\
\textit{Alla tyskar jagar älg} &\\
\end{tabular}
\end{table}

\newpage
\setlength{\oddsidemargin}{-0.37in}
\begin{table}[!h]
\begin{tabular}{l l}
\textit{Et langue d'or joue haute fadeur}	&$\delta$ 1c\\
\textit{a l'an la langue laide!} &\\
\textit{Festen skall börjas,}	&$\kappa$ 1\\
\textit{kråset ska smörjas} &\\
\textit{Feta fransyskor som}	&$\varepsilon$ 2\\
\textit{svettas om fötterna} &\\
\textit{Fkåne faft}	&$\delta$ 5a\\
\textit{Flamma stolt mot dunkla skyar}	&$\alpha$ 3\\
\textit{fot kilopond Kayser bushel Beaufort}	&$\iota$ 17\\
\textit{För att människan ska}	&$\delta$ 5d\\
\textit{trivas på vår jord} &\\
\textit{För det var i vår ungdoms fagraste vår}	&$o$ 2\\
\textit{Från Öckerö loge hörs dragspel och bas}	&$\kappa$ 9\\
\textit{Full och galen med moralen minimal}	&$\delta$ 7b\\
\textit{Gammalt brännvin, gammalt brännvin}	&$\delta$ 15c\\
\textit{Gaudeamus igitur, iuvenes dum sumus!}	&$o$ 3\\
\textit{Glad åt en fågel i morgonstunden}	&$\nu$ 6\\
\textit{går hen på jakt i den friska natur} &\\
\textit{Glenlivet, Highland park, Glegoyne}	&$\eta$ 12\\
\textit{Goda vänner, låt oss fatta glaset}	&$\varepsilon$ 4\\
\textit{Gräv ur tundran två dussin potäter}	&$\delta$ 8d\\
\textit{Här är gudagott att vara}	&$\lambda$ 10\\
\textit{Här i Bayern ska ölet flöda fritt}	&$\gamma$ 1\\
\textit{Här på festen stiger åter glammet}	&$o$ 1\\
\textit{Här på skeppet rinner rommen}	&$\eta$ 3\\
\textit{Här ska ni allt få chockeras}	&$\varepsilon$ 3\\
\textit{Hej på er vänner alla}	&$\beta$ 7\\
\textit{Hej, tomtegubbar, vrid på gasen}	&$\delta$ 14d\\
\textit{Helan går, sjung hopp faderallan...}	&$\delta$ 1a\\
\textit{Helan gick  vänstra foten}	&$\delta$ 2d\\
\textit{Helan rasat ned i våra magar}	&$\delta$ 2e\\
\end{tabular}
\end{table}

\newpage
\setlength{\oddsidemargin}{-0.47in}
\begin{table}[!h]
\begin{tabular}{l l}
\textit{Hell and Gore, Chung Hop father Allan...}	&$\delta$ 1b\\
\textit{Hur badar man bäst på en kurort?}	&$\varepsilon$ 9\\
\textit{Hur gärna skulle jag ej vara}	&$\kappa$ 7\\
\textit{en liten blå förgätmigej} &\\
\textit{Hur länge skall på borden den lilla halvan stå?}	&$\delta$ 2b\\
\textit{Hurra nu ska en äntligen få röra på benen}	&$\beta$ 3\\
\textit{Huvet slåt kopparslag,}	&$\delta$ 13d\\
\textit{ögonen svider} &\\
\textit{Huvudet vi lyfter med}	&$\theta$ 5\\
\textit{ett stön ur vår säng} &\\
\textit{I Indialand, bak Himalayas rand}	&$\nu$ 11\\
\textit{I natt jag drömde något som...}	&$\gamma$ 2\\
\textit{I Spanien hör det till}	&$\eta$ 1\\
\textit{god ton och etikett} &\\
\textit{Ikväll har Napoleon}	&$\zeta$ 1/$\varepsilon$\\
\textit{gjort England den äran} &\\
\textit{I'm a lumberjack and I'm OK}	&$\nu$ 13\\
\textit{Imbelupet glaset står på bräcklig fot}	&$\delta$ 1d\\
\textit{Ingen har det så bra som jag}	&$\delta$ 16a\\
\textit{Integralkalkylens fader}	&$\iota$ 2\\
\textit{G W Leibniz hette han} &\\
\textit{Ja, brännvin är jävligt gott}	&$\delta$ 11d\\
\textit{Jag är en liten teknolog}	&$\beta$ 10\\
\textit{Jag är en liten undulat,}	&$\nu$ 7\\
\textit{som får dåligt med mat} &\\
\textit{Jag är teknolog och helt OK}	&$\mu$ 7\\
\textit{Jag drömmer om en vit vecka}	&$\theta$ 8\\
\textit{Jag fångade en räv idag}	&$\delta$ 6d\\
\textit{Jag fångade en sup idag}	&$\delta$ 6e\\
\textit{Jag gillar alla tiders punsch}	&$\zeta$ 7\\
\end{tabular}
\end{table}

\newpage
\setlength{\oddsidemargin}{-0.37in}
\begin{table}[!h]
\begin{tabular}{l l}
\textit{Jag gillar inte höghus,}	&$\nu$ 12\\
\textit{sten och lätt betong} &\\
\textit{Jag gillar, jag gillar meken} &$\iota$ 18\\
\textit{Jag har aldrig vart på snusen}	&$\mu$ 1\\
\textit{aldrig rökat en cigarr, helelujah!} &\\
\textit{som finns att pröva på} &\\
\textit{Jag har druckit många punschar}	&$\beta$ 11\\
\textit{Jag har prövat nästan allt}	&$\iota$ 9\\
\textit{Jag har prövat nästan allt }	&$\mu$ 6\\
\textit{som finns att välja på} &\\
\textit{Jag heter Göran Grimvall, W pdV Vdp}	&$\iota$ 3\\
\textit{Jag heter Patrik, ja det är mig}&$\iota$ 19\\
\textit{Jag minns än i dag hur min fader}	&$\varepsilon$ 7\\
\textit{Jag minns knappt hur jag tog mig hem}	&$\nu$ 15\\
\textit{Jag skall festa, ta det lugnt med spriten}	&$\beta$ 5\\
\textit{Jag uppstämma vill min lyra}	&$\nu$ 3\\
\textit{fast det blott är en gitarr} &\\
\textit{Jag var full en gång för länge se'n}	&$\theta$ 2\\
\textit{Jag var ung konduktör}	&$\varepsilon$ 10\\
\textit{Jag vill aldrig gå på Handels,}	&$\mu$ 2\\
\textit{aldrig tenta företagsekonomi} &\\
\textit{Jag vill börja gasqua,}	&$\beta$ 1\\
\textit{var fan är min flaska?} &\\
\textit{Jag vill inte gå på fysik,}	&$\mu$ 3\\
\textit{aldrig tenta termometerdynamik} &\\
\textit{Jag vill inte ha - \textbf{en nubbe till!}}	&$\delta$ 9d\\
\textit{Ju mera öl vi dricker, vi dricker}	&$\gamma$ 10\\
\textit{Kaffe, kaffe, kaffe, konjak och likör}	&$\eta$ 10\\
\textit{Kalla den änglamarken}	&$\kappa$ 4\\
\textit{eller himlajorden om du vill} &\\
\textit{Kantom studjosi extarbon sjur!}	&$\nu$ 10\\
\end{tabular}
\end{table}

\newpage
\setlength{\oddsidemargin}{-0.47in}
\begin{table}[!h]
\begin{tabular}{l l}
\textit{Kom, du ljuva hjärtevän!}	&$\lambda$ 9\\
\textit{Skall jag vänta länge än?} &\\
\textit{Kristallen den fina, som solen månd' skina}	&$\lambda$ 8\\
\textit{Krök armen i vinkel, här vankas det finkel}	&$\delta$ 9c\\
\textit{Länge har jag tänkt att punschen övergiva}	&$\zeta$ 5\\
\textit{Lapin Kulta, Lapin Kulta}	&$\gamma$ 9\\
\textit{Lille olle skulle gå på disco}	&$\beta$ 8\\
\textit{Livet är härligt! Tavaritj, vårt liv är härligt!}	&$\delta$ 8a\\
\textit{Lyft ditt välförsedda glas}	&$\beta$ 2\\
\textit{Magen brummar. Jag försummar...}	&$\delta$ 15b\\
\textit{Min pilsner skall svalka min tunga}	&$\gamma$ 4\\
\textit{Min resa var mot solen} &$\kappa$ 15\\
\textit{Minne, jag har tappat mitt minne!}	&$\theta$ 4\\
\textit{Mitt namn är Nikolajev,}	&$\nu$ 4\\
\textit{kosmonaut från Sovjet} &\\
\textit{Mjölk, mjölk vi vill ha mjölk}	&$\eta$ 5\\
\textit{Mohrs lilla cirkel i skogen kröp}	&$\iota$ 10\\
\textit{Morgonstund med smak av döda bävrar}	&$\theta$ 7\\
\textit{Mors lilla dator åt skogen gick}	&$\iota$ 6\\
\textit{Morsgrisar små ska inte supa}	&$\delta$ 7e\\
\textit{När det strålar uti salen}	&$\varepsilon$ 1\\
\textit{När helan en tagit}	&$\delta$ 2a\\
\textit{När jag fuller, då är jag glad}	&$\delta$ 10g\\
\textit{När jag tar mig en sup}	&$\delta$ 17b\\
\textit{blir jag intressant och djup} &\\
\textit{När månen vandrar sin tysta ban}	&$\delta$ 10a\\
\textit{När nubben blänker i immigt glas}	&$\delta$ 10e\\
\textit{När punschen så småningom ta't slut}	&$\zeta$ $\infty$\\
\textit{När snapsen vandrat hädan}	&$\zeta$ 2\\
\textit{och maten lagts därpå} &\\
\end{tabular}
\end{table}

\newpage
\setlength{\oddsidemargin}{-0.37in}
\begin{table}[!h]
\begin{tabular}{l l}
\textit{När som sädesfälten böja sig för vinden}	&$\delta$ 6c\\
\textit{Nu är det dags att taga supen}	&$\delta$ 5b\\
\textit{Nu grönskar det i dalens famn}	&$\lambda$ 6\\
\textit{Nu har alla lämnat festen}	&$\theta$ 1\\
\textit{och maten lagts därpå} &\\
\textit{Nu skall vi klämma septen gutår}	&$\delta$ 7a\\
\textit{NU!!!}	&$\delta$ 6a\\
\textit{O gamla klang och jubeltid,}	&$o$ $\infty$\\
\textit{ditt minne skall förbliva} &\\
\textit{O hemska lab, o fasliga arbete}	&$\iota$ 12c\\
\textit{O hemska lab, o grymma kval imorgon}	&$\iota$ 12a\\
\textit{O hemska lab, o grymma kval imorgon}	&$\iota$ 12b\\
\textit{Och nubben kallas också göken}	&$\delta$ 15a\\
\textit{Oh don't give me none more}	&$\delta$ 14c\\
\textit{of that Old Janx Spirit} &\\
\textit{Opp och hoppa, Tor.}	&$\eta$ 8\\
\textit{Slå på trumman, bror} &\\
\textit{Punsch, punsch filibombombom}	&$\zeta$ 9\\
\textit{Punsch, punsch,punsch,}	&$\zeta$ 6\\
\textit{mera punsch, mera punsch!} &\\
\textit{Punschen är och punschen var}	&$\zeta$ 4\\
\textit{och punschen skall förbliva} &\\
\textit{Punschen kommer, punschen kommer}	&$\zeta$ 1\\
\textit{Punschen, punschen,}	&$\zeta$ 11\\
\textit{rinner genom strupen} &\\
\textit{Ren som en jomfru og}	&$\kappa$ 11\\
\textit{stærk som en bejler} &\\
\textit{Röda vinet, röda vinet, uti glasen står}	&$\varepsilon$ 11a\\
\textit{Rönnerdahl han skuttar}	&$\kappa$ 3\\
\textit{med ett skratt ur sin säng} &\\
\textit{Runt kring vår stuga smådjävlar sluga...}	&$\nu$ 5\\
\end{tabular}
\end{table}
\newpage
\setlength{\oddsidemargin}{-0.47in}
\begin{table}[!h]
\begin{tabular}{l l}
\textit{Så hastigt den lilla nubben}	&$\delta$ 8c\\
\textit{i strupen försvann} &\\
\textit{Så länge rösten är mild}	&$\varepsilon$ 8\\
\textit{Så lunka vi så småningom}	&$\lambda$ 1\\
\textit{Samborombon, en liten by förutan gata}	&$\kappa$ 5\\
\textit{Se hur hela Teknis går i vånda}	&$\iota$ 7\\
\textit{Selen lever, i vår hånd}	&$\theta$ 12\\
\textit{Should auld acquaintance be forgot}	&$\alpha$ 5\\
\textit{Sjung om fru Svenssons lyckliga karl}	&$\nu$ 17\\
\textit{Sjung om studentens lyckliga dag!}	&$o$ 5\\
\textit{Skål, kamrater, livet är glatt}	&$\delta$ 4c\\
\textit{Små nubbarna, små nubbarna}	&$\delta$ 3c\\
\textit{Smuttans ungar har just runnit ner}	&$\delta$ 12b\\
\textit{Solen den går upp och ner,}	&$\delta$ 14a\\
\textit{doda, doda!} &\\
\textit{Solen går upp och ner,}	&$\delta$ 14b\\
\textit{snapsen den går ner!} &\\
\textit{Solen glimmar blank och trind,}	&$\lambda$ 3\\
\textit{vattnet likt en spegel} &\\
\textit{Spelaren drog fiol'n ur lådan}	&$\kappa$ 10\\
\textit{och lyfte stråken högt} &\\
\textit{Stackars styrman Karlsson hade otur}	&$\nu$ 16\\
\textit{Ström, ström, vi vill ha ström}	&$\iota$ 11\\
\textit{Täckt av silver sejdeln full}	&$\gamma$ 11\\
\textit{Tänk att tentera reglerteknik, lilla jag}&$\iota$ 21\\
\textit{Tänk om jag hade lilla nubben...}	&$\delta$ 9b\\
\textit{Teknisk fysik är mössbeklädda töntar}	&$\mu$ 5\\
\textit{Teknologen vaknar upp en morgon}    &$\mu$ 8\\
\end{tabular}
\end{table}
\newpage
\setlength{\oddsidemargin}{-0.37in}
\begin{table}[!h]
\begin{tabular}{l l}
\textit{Temperaturen är hög uti kroppen}	&$\nu$ 37,5\\
\textit{Tenn arsenik bor platina brom}	&$\iota$ 16\\
\textit{Till spritbutiken ränner jag}	&$\beta$ 6\\
\textit{Till supen så tager en sill, sill, sill}	&$\delta$ 13b\\
\textit{Tjugotre är Bäska Droppar}	&$\delta$ 16b\\
\textit{Toj hemtegubbar gla i slåsen}	&$\delta$ 7c\\
\textit{Törsten rasar uti våra strupar}	&$\delta$ 4a\\
\textit{Trink, trink, Brüderlein trink}	&$\kappa$ 12\\
\textit{Trrrrrretton kärringar}	&$\delta$ 13e\\
\textit{i Lundströms kök} &\\
\textit{Tu-tu-tu Tuborg}	&$\gamma$ 3\\
\textit{och ca-ca-ca Carlsberg} &\\
\textit{Två studenter på teknis}	&$\kappa$ 14\\
\textit{satt i skolan hand i hand} &\\
\textit{Under en filt i Madrid}	&$\nu$ 14\\
\textit{där ligger en flicka på glid} &\\
\textit{Upp och ner och charm och sär}	&$\iota$ 14\\
\textit{Upp trälar uti alla stater}	&$\alpha$ 4\\
\textit{Uppå bordet står nu en liten tår}	&$\zeta$ 3\\
\textit{Ur svenska hjärtans djup en gång}	&$\alpha$ 2\\
\textit{Ur svenska hjärtans djup en sup}	&$\delta$ 12a\\
\textit{Ur teknologens djup en gång,}	&$o$ 4\\
\textit{en samfälld och en enkel sång} &\\
\textit{Ut från D1 rusade en fysiker en gång} &$\iota$ 23\\
\textit{Uti Kalmare stad ja}	&$\beta$ 4\\
\textit{där finns ingen kvast} &\\
\textit{Uti min mage en längtan mig tär}	&$\delta$ 17a\\
\textit{Uti vår hage där växa blå bär}	&$\lambda$ 12\\
\textit{Vad är det som gör}	&$\delta$ 11g\\
\textit{att en skojare trivs?} &\\
\textit{Vad dom kan göra i vårt mejeri}	&$\eta$ 6\\
\end{tabular}
\end{table}
\newpage
\setlength{\oddsidemargin}{-0.47in}
\begin{table}[!h]
\begin{tabular}{l l}
\textit{Var är absinten}	&$\eta$ 7\\
\textit{illgrön liksom minten} &\\
\textit{Var Osquristina som}	&$\delta$ 3b\\
\textit{går på vår sektion} &\\
\textit{Var redo, var redo,}	&$\delta$ 13a\\
\textit{för nu skall supen tas} &\\
\textit{Vårvindar friska, leka och viska}	&$\lambda$ 4\\
\textit{Vem kan hugga sig själv i knät?}	&$\delta$ 12e\\
\textit{Vem kan kröka förutan krök?}	&$\delta$ 12d\\
\textit{Vem sade ordet "skål" här vid bordet?}	&$\delta$ 1e\\
\textit{Vi, våran verklighet har gått i kras}&$\iota$ 20\\
\textit{Vi äro små änglar vi, flax, flax}	&$\delta$ 11c\\
\textit{Vi äro små fiskar vi, blubb, blubb}	&$\delta$ 11b\\
\textit{Vi äro små humlor vi, bzzz, bzzz}	&$\delta$ 11a\\
\textit{Vi dricka, vi dricka}	&$\theta$ 9\\
\textit{upp allt som dukas fram} &\\
\textit{Vi dricker öl, vi dricker alkohol}	&$\beta$ 9\\
\textit{Vi dricker punsch till lunch}	&$\zeta$ 10\\
\textit{Vi går över ån efter sprit, fallera!}	&$\delta$ 13c\\
\textit{Vi har ägnat våren}	&$\iota$ 8\\
\textit{åt en kurs i mekanik} &\\
\textit{Vi kan dricka Sädes}	&$\delta$ 9a\\
\textit{och vi kan dricka Kron} &\\
\textit{Vi ska röja, vi ska härja}	&$\theta$ 11b\\
\textit{Vi ska supa, vi ska fest}	&$\theta$ 11a\\
\textit{Vi söker våra rötter,}	&$\eta$ 13\\
\textit{dricker allt som rött är} &\\
\textit{Vi som oss för att glupa satt,}	&$\delta$ 7d\\
\textit{supa glatt} &\\
\textit{Vi vill ha mera järn}	&$\delta$ 12c\\
\end{tabular}
\end{table}
\newpage
\setlength{\oddsidemargin}{-0.37in}
\begin{table}[!]
\begin{tabular}{l l}
\textit{Vi vill ha punsch (knäpp, knäpp)}	&$\zeta$ 8\\
\textit{Vinet väntar, vinet väntar}	&$\varepsilon$ 11b\\
\textit{Vintern rasat ut bland våra fjällar}	&$\kappa$ 8\\
\textit{Vodka, vodka vill jag dricka}	&$\delta$ 8b\\
\textit{Watt kilogram meter weber sekund}	&$\iota$ 1\\
\textit{When I get drunker,}	&$\theta$ 10\\
\textit{loosing my mind} &\\
\textit{Whiskyn är förädling utav ölet}	&$\eta$ 11\\
\textit{Yoghurt, yoghurt,}	&$\eta$ 9\\
\textit{fyller oss till brädden} &\\
\textit{Å så kommer det en ångbåt}	&$\delta$ 1f\\
\textit{Å så kommer det en geting}	&$\delta$ 11f\\
\textit{Årets första fest}	&$\kappa$ 2\\
\textit{sker på klassiskt manér} &\\
\textit{Än en gång däran, vänner,}	&$\kappa$ 6\\
\textit{än en gång däran} &\\
\textit{Är du trött på att va' som andra,}	&$\nu$ 2\\
\textit{driva med i livets älv?} &\\
\textit{Öl är gudagott att dricka}	&$\gamma$ 5\\
\textit{Öl, öl, öl i glas eller i butelj}	&$\gamma$ 8\\ 
\vspace{150pt}
\end{tabular}
\end{table}
\end{document}