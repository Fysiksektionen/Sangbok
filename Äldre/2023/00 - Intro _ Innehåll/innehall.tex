\documentclass[a6paper,10pt]{article}
\usepackage[T1]{fontenc}
\usepackage[british]{babel}
\usepackage[utf8]{inputenc}
\usepackage{float, graphicx,amsmath,amsfonts,cite,enumerate,tabularx}
\usepackage[final]{pdfpages}
\usepackage{wrapfig}
\usepackage[margin=0.3in]{geometry}
\usepackage{sidspaltHack}
\newcommand{\mel}[1]{\small\textbf{\textit{mel. #1 \\}}}

\setlength{\oddsidemargin}{-0.37in}
\setlength{\textwidth}{225pt}

\pagestyle{empty}

\begin{document}
\nysida{0}{}
\noindent \huge Innehåll

\vspace{5pt}

\noindent \Large Etikett 

\vspace{5pt}

\noindent \Large $\alpha$ (alfa) - Visor till högtid

\normalsize
\noindent
\begin{tabularx}{0.55\textwidth}{l X}
1.&Du gamla du fria\\
2.&Kungssången\\
3.&Sveriges flagga\\
4.&Internationalen\\
5.&Auld lang syne
\end{tabularx}

\vspace{5pt}

\noindent \Large Gasqueregler

\vspace{5pt}

\noindent \Large $\beta$ (beta) - Visor till gasque

\noindent
\normalsize
\begin{tabularx}{0.65\textwidth}{l X}
1.&Porthos visa\\
2.&Lyft ditt välförsedda glas\\
3.&Härjavisan\\
4.&Kalmarevisan\\
5.&Jag skall festa \\
6.&Emils spritvisa\\
7.&Hej på er vänner alla\\
8.&Lille Olle\\
9.&Gasqueljäsen\\
10.&Mattevisan\\
11.&37:an
\end{tabularx}

\newpage

\noindent \Large $\gamma$ (gamma) - Visor till öl

\noindent
\normalsize
\begin{tabularx}{0.65\textwidth}{l X}
1.&Ölbytarvisan\\
2.&Strejk på Pripps\\
3.&Ode till ölet\\
4.&Min pilsner\\
5.&Sanningen om ölet \\
6.&Ölvisan\\
7.&En pilsnerdrickare\\
8.&Öl, öl, öl i glas\\
9.&Lapin Kulta\\
10.&Ju mera öl vi dricker\\
11.&Vi älskar öl
\end{tabularx}

\vspace{5pt}

\noindent \Large $\delta$ (delta) - Visor till destillat

\normalsize

\vspace{3pt}

\noindent Supregler

\vspace{3pt}

\noindent De sjutton suparna intagande

\vspace{3pt}

\noindent
\begin{tabularx}{0.8\textwidth}{l X}
1. Helan&Helan går\\
&Hell and Gore\\
&Et langue d'or\\
&Imbelupet\\
&Vem sade ordet "skål"?\\
&Ubåten\\
&Heltal rho\\
2. Halvan&När helan en tagit\\
&Halvan\\
&Angorakatten\\
&Helan gick\\
&Helan rasat
\end{tabularx}

\noindent
\begin{tabularx}{0.9\textwidth}{l X}
3. Tersen&Nubbekantat\\
 &Var Osquristina\\
 &Små nubbarna\\
4. Kvarten&Planksaft\\
 &Brännvin hit\\
 &Gums visa\\
5. Kvinten&Fkåne faft\\
 &Mera Skåne\\
 &En cyklar för lite\\
 &För att människan\\
6. Rivan&Finsk snapsvisa\\
 &Finsk brännvinsvisa\\
 &Sädesfälten\\
 &Räven\\
 &Supen\\
7. Septen&Nu ska vi klämma septen\\
 &Full och galen\\
 &Toj hemtegubbar\\
 &Full är bäst\\
 &Morsgrisar små\\
8. Rafflan&Livet är härligt\\
 &Vodka, vodka\\
 &Så hastigt\\
 &Gräv ur tundran\\
9. Rännan&Hyllning till OP Andersson\\
&Tänk om jag hade lilla nubben\\
&Krök armen\\
&Inre dialog\\
10. Smuttan&Månvisa\\
&Måsen
\end{tabularx}

\newpage

\noindent
\begin{tabularx}{1\textwidth}{l X}
&Den vingklippta måsen\\
&JASen\\
&När nubben blänker\\
&Moose:en\\
&När jag är fuller\\
&Mesen\\
&Musen\\
&Måsens sista sup\\
11. Smuttans&Humlorna\\
\hspace{17pt}ungar&Fiskarna\\
&Änglarna\\
&Brännvin är jävligt gott\\
&En liten fyllhund\\
&Getingen\\
&Kalla små nubbar\\
12. Femton&Mod i barm\\
\hspace{17pt}droppar&O, besinna\\
&Mera järn\\
&Vemkan kröka\\
&Vem kan hugga\\
13. Lilla&Var redo!\\
\hspace{17pt}Manasse&Till supen så tager en sill\\
&Vi går över ån\\
&Sänkta Lucia\\
&Lundströms kök\\
14. Lilla&Solen\\
\hspace{17pt}Manasses&Korta solen\\
\hspace{17pt}bror&Old Janx Spirit\\
&Hörapparaten\\
15. Kreaturens&Göken\\
\hspace{17pt}uppståndelse&Magen brummar
\end{tabularx}
\begin{tabularx}{1\textwidth}{l X}
&Gammalt brännvin\\
&Månen (En gång i månan)\\
16. Absolut sista&Raj-raj\\
\hspace{17pt}supen&Tjugotre\\
&Vikingen\\
&Feministvikingen\\
&Nykteristvikingen\\
17. Den bleka&Uti min mage\\
\hspace{17pt}dödens dryck&Då verka lätt
\end{tabularx}

\vspace{5pt}

\noindent \Large $\varepsilon$ (epsilon) - Visor till vin

\normalsize
\noindent
\begin{tabularx}{1\textwidth}{l X}
1.&Vinets lov\\
2.&Feta fransyskor\\
3.&Vinvisa (Har ni sett på attan)\\
4.&Vinet skänker\\
5.&Du gamla vin\\
6.&Elysisk längtan\\
7.&Bordeaux, Bordeaux\\
8.&Spegelvisa\\
9.&Röd vitamin\\
10.&Portvinsvisa\\
11.&Röda vinet\\
&Vinet väntar
\end{tabularx}

\vspace{5pt}

\noindent \Large $\zeta$ (zeta) - Visor till punsch

\normalsize
\noindent
\begin{tabularx}{1\textwidth}{l X}
1.&Punschen kommer\\
2.&Punschkanon\\
3.&Punschschottis\\
4.&Punschens lov\\
5.&Jag gillar punschen\\
6.&Imperial punsch\\
\end{tabularx}

\newpage

\noindent
\begin{tabularx}{1\textwidth}{l X}
7.&Djungelpunsch\\
8.&Vi vill ha punsch\\
9.&Punsch, punsch\\
10.&Studiemedelsrondo\\
11.&Flaggpunschens visa\\
12.&Visa en torsdagskväll\\
13.&Sveriges Arraktionalhymn\\
1/$\varepsilon$.&Punschfinalen\\
$\infty$.&Sista punschvisan
\end{tabularx}

\vspace{5pt}

\noindent \Large $\eta$ (eta) - Visor till andra drycker

\normalsize
\noindent
\begin{tabularx}{1\textwidth}{l X}
1.&Skål för vattnet\\
2.&En kan dricka vatten\\
3.&Nu tar vi rom\\
4.&Däj-o\\
5.&Mjölk\\
6.&Mjölksång\\
7.&Hyllningsvisa till absinten\\
8.&Schottis på Valhall\\
9.&Häflåten\\
10.&Kaffe\\
11.&Whiskyn\\
12.&1, 2, 3, Whisky!\\
13.&Kahlua
\end{tabularx}

\vspace{5pt}

\noindent \Large $\theta$ (theta) - Blöta visor

\noindent \normalsize Salomos ordspråk

\vspace{5pt}

\noindent
\begin{tabularx}{1\textwidth}{l X}
1.&Störthärligt full\\
2.&Jag var full en gång\\
3.&Bär ner mig till sjön
\end{tabularx}

\noindent
\begin{tabularx}{1\textwidth}{l X}
4.&Minne\\
5.&Antisnapsvisa\\
6.&Dom som är nyktra\\
7.&Treo-comp\\
8.&Vit vecka\\
9.&Vi dricka, vi dricka\\
10.&When I get drunker\\
11.&Vi ska supa\\
&Härjarens bordsvisa\\
12.&Selen lever
\end{tabularx}

\vspace{5pt}

\noindent \Large $\iota$ (iota) - Torra Visor

\noindent
\normalsize
\begin{tabularx}{1\textwidth}{l X}
1.&Système International\\
2.&Integralkalylens fader\\
3.&GG-visan\\
4.&Öl sex\\
5.&The BASIC song\\
6.&Mors lilla dator\\
7.&Tentamenssång\\
8.&ODE till en husvagn\\
9.&Matlab\\
10.&Hållfvisa\\
11.&Elämnenas lov\\
12.&O hemska lab\\
&O hemska lab\\
&O hemska lab\\
13.&Aris summavisa\\
14.&Kvarkvisan\\
15.&Liten visa om Gram-Schmidts metod\\
\end{tabularx}

\noindent
\begin{tabularx}{1\textwidth}{l X}
16.&Kemisången\\
17.&Imperial system\\
18.&Jag gillar meken\\
19.&Termon\\
20.&Henelius-eufori\\
21.&Reglerteknik på bal\\
22.&En matematiker\\
23.&Det är långt bort till Alba Nova\\
24.&Tentapluggsblues\\
25.&Système Bolaget
\end{tabularx}

\vspace{5pt}

\noindent \Large $\kappa$ (kappa) - Fina Visor

\noindent
\normalsize
\begin{tabularx}{1\textwidth}{l X}
1.&Festen skall börjas\\
2.&Festvisa\\
3.&Sjösala vals\\
4.&Änglamark\\
5.&Fritiof och Carmencita\\
6.&Än en gång däran\\
7.&En liten blå förgätmigej\\
8.&Längtan till landet\\
9.&Balladen om Herr Fredrik Åkare och den söta fröken Cecilia
Lind\\
10.&Hårgalåten\\
11.&En dansk aquavit\\
12.&Tring, trink\\
13.&Smedsvisan\\
&Korta smedsvisan\\
14.&Balladen om ett kärlekspar\\
15.&Stad i ljus
\end{tabularx}

\newpage

\noindent \Large $\lambda$ (lambda) - Vackra Visor

\noindent
\normalsize
\begin{tabularx}{1\textwidth}{l X}
1.&Fredmans sång n:o 21 - Måltidssång\\
2.&Kor ur Bacchi tempel\\
3.&Fredmans epistel n:o 48\\
4.&Molltoner från Norrland\\
5.&Den blomstertid\\
6.&Nu grönskar det \\
7.&Visa vid midsommartid\\
8.&Kristallen den fina\\
9.&Madrigal\\
10.&Glunt nr 25 - Examenssexa\\
11.&Sommarpsalm\\
12.&Uti vår hage\\
13.&Värmlandsvisan
\end{tabularx}

\vspace{5pt}

\noindent \Large $\mu$ (my) - Nidvisor

\noindent
\normalsize
\begin{tabularx}{1\textwidth}{l X}
1.&Jag har aldrig vart på snusen\\
2.&Handelsvisa\\
3.&Fysikhatarvisan\\
$\pi$.&Matematikhatarvisan\\
4.&Datas visa\\
6.&Hyllningsvisa\\
7.&Man ska gå teknis \\
8.&Teknologvisa\\
9.&Teknologen och ekonomen
\end{tabularx}

\vspace{5pt}

\noindent \Large $\nu$ (ny) - Skojiga Visor

\noindent
\normalsize
\begin{tabularx}{1\textwidth}{l X}
1.&Lingonben\\
2.&Älska dig själv\\
3.&Balladen om den kaxiga myran\\
\end{tabularx}

\noindent
\begin{tabularx}{1\textwidth}{l X}
4.&Nikolajev\\
5.&Danse Macabre\\
6.&Katten \\
7.&Undulaten \\
8.&Bakfyllosofen\\
9.&Mellansup\\
10.&Kanta Studjosi\\
11.&Indialand\\
12.&Zwampen\\
13.&Lumberjack song\\
14.&Under en filt i Madrid\\
15.&Hallen luta\\
16.&Styrman Karlssons äventyr med porslinspjäsen\\
17.&Sjung om Fru Svenssons lyckliga karl\\
18.&Jesus lever\\
37,5.&Temperaturen
\end{tabularx}

\vspace{5pt}

\noindent \Large $o$ (omikron) - Visor till Fysiker

\noindent
\normalsize
\begin{tabularx}{1\textwidth}{l X}
1.&Konglig Fysiks Paradhymn\\
2.&Årskursernas hederssång\\
3.&De Brevitate Vitae (Gaudeamus)\\
4.&Överföhssången\\
5.&Studentsången\\
6.&När Fumla blev fem\\
$\infty$.&O gamla klang och jubeltid
\end{tabularx}

\vspace{5pt}

\noindent \Large $\sigma$ (sigma) - Visor vi minns

\newpage

\noindent \Large $\omega$ (omega) - Noter

\noindent
\normalsize
\begin{tabularx}{1\textwidth}{l X}
1.&Du gamla du fria\\
2.&Kungssången\\
3.&Sveriges flagga\\
4.&Porthos visa\\
5.&Lyft ditt välförsedda glas\\
6.&Längtan till landet\\
7.&Amanda Lundbom\\
8.&Smedsvisa\\
9.&Molltoner från Norrland\\
10.&Nu grönskar det\\
11.&Studentsången\\
$\infty$.&O gamla klang och jubeltid
\end{tabularx}

\vspace{5pt}

\noindent
\Large Register

\end{document}