\documentclass[a6paper,10pt]{article}
%\usepackage[T1]{fontenc}
\usepackage[british]{babel}
\usepackage[utf8]{inputenc}
\usepackage{float, graphicx,amsmath,amsfonts,cite,enumerate,tabularx}
\usepackage[final]{pdfpages}
\usepackage{wrapfig}
\usepackage[margin=0.3in]{geometry}
\usepackage{sidspaltHack}
\newcommand{\mel}[1]{\small\textbf{\textit{mel. #1 \\}}}


\setlength{\oddsidemargin}{-0.37in}
\setlength{\textwidth}{225pt}

\pagestyle{empty}

\begin{document}
\nysida{0}{}
\noindent
\huge{Innehåll}
\vspace{5pt}\\
\Large Etikett 
\vspace{5pt}\\
\Large $\alpha$ (alfa) - Visor till högtid
\vspace{-10pt}
\begin{table}[!ht]
\begin{tabularx}{0.55\textwidth}{l X}
1.&Du gamla du fria\\
2.&Kungssången\\
3.&Sveriges flagga\\
4.&Internationalen\\
5.&Auld lang syne
\end{tabularx}
\end{table}
\vspace{-5pt}\\
\Large Gasqueregler
\vspace{5pt}\\
\Large $\beta$ (beta) - Visor till gasque
\vspace{-10pt}
\begin{table}[!ht]
\begin{tabularx}{0.65\textwidth}{l X}
1.&Porthos visa\\
2.&Lyft ditt välförsedda glas\\
3.&Härjavisan\\
4.&Kalmarevisan\\
5.&Jag skall festa \\
6.&Emils spritvisa\\
7.&Hej på er vänner alla\\
8.&Lille Olle\\
9.&Gasqueljäsen\\
10.&Mattevisan\\
11.&37:an
\end{tabularx}
\end{table}

\newpage
\setlength{\oddsidemargin}{-0.47in}
\noindent
\Large $\gamma$ (gamma) - Visor till öl
\vspace{-10pt}
\begin{table}[!ht]
\begin{tabularx}{0.65\textwidth}{l X}
1.&Ölbytarvisan\\
2.&Strejk på Pripps\\
3.&Ode till ölet\\
4.&Min pilsner\\
5.&Sanningen om ölet \\
6.&Ölvisan\\
7.&En pilsnerdrickare\\
8.&Öl, öl, öl i glas\\
9.&Lapin Kulta\\
10.&Ju mera öl vi dricker\\
11.&Vi älskar öl
\end{tabularx}
\end{table}

\noindent
\Large $\delta$ (delta) - Visor till destillat
\vspace{3pt}\\
\normalsize   Supregler
\vspace{3pt}\\
    De sjutton suparna intagande
\vspace{-7pt}
\begin{table}[!ht]
\begin{tabularx}{0.8\textwidth}{l X}
1. Helan&Helan går\\
&Hell and Gore\\
&Et langue d'or\\
&Imbelupet\\
&Vem sade ordet "skål"?\\
&Ubåten\\
2. Halvan&När helan en tagit\\
&Halvan\\
&Angorakatten\\
&Helan gick\\
&Helan rasat\\
\end{tabularx}
\end{table}
\begin{table}[!ht]
\begin{tabularx}{0.9\textwidth}{l X}
3. Tersen&Nubbekantat\\
 &Var Osquristina\\
 &Små nubbarna\\
4. Kvarten&Planksaft\\
 &Brännvin hit\\
 &Gums visa\\
5. Kvinten&Fkåne faft\\
 &Mera Skåne\\
 &En cyklar för lite\\
 &För att människan\\
6. Rivan&Finsk snapsvisa\\
 &Finsk brännvinsvisa\\
 &Sädesfälten\\
 &Räven\\
 &Supen\\
7. Septen&Nu ska vi klämma septen\\
 &Full och galen\\
 &Toj hemtegubbar\\
 &Full är bäst\\
 &Morsgrisar små\\
8. Rafflan&Livet är härligt\\
 &Vodka, vodka\\
 &Så hastigt\\
 &Gräv ur tundran\\
9. Rännan&Hyllning till OP Andersson\\
&Tänk om jag hade lilla nubben\\
&Krök armen\\
&Inre dialog\\
10. Smuttan&Månvisa\\
&Måsen
\end{tabularx}
\end{table}
\begin{table}[!ht]
\begin{tabularx}{1\textwidth}{l X}
&Den vingklippta måsen\\
&JASen\\
&När nubben blänker\\
&Moose:en\\
&När jag är fuller\\
&Mesen\\
&Musen\\
&Måsens sista sup\\
11. Smuttans&Humlorna\\
\hspace{17pt}ungar&Fiskarna\\
&Änglarna\\
&Brännvin är jävligt gott\\
&En liten fyllhund\\
&Getingen\\
&Kalla små nubbar\\
12. Femton&Mod i barm\\
\hspace{17pt}droppar&O, besinna\\
&Mera järn\\
&Vemkan kröka\\
&Vem kan hugga\\
13. Lilla&Var redo!\\
\hspace{17pt}Manasse&Till supen så tager en sill\\
&Vi går över ån\\
&Sänkta Lucia\\
&Lundströms kök\\
14. Lilla&Solen\\
\hspace{17pt}Manasses&Korta solen\\
\hspace{17pt}bror&Old Janx Spirit\\
&Självmördarvisan\\
&Hörapparaten\\
15. Kreaturens&Göken\\
\hspace{17pt}uppståndelse&Magen brummar
\end{tabularx}
\end{table}
\newpage
\begin{table}[!ht]
\begin{tabularx}{1\textwidth}{l X}
&Gammalt brännvin\\
&Månen (En gång i månan)\\
16. Absolut sista&Raj-raj\\
\hspace{17pt}supen&Tjugotre\\
&Vikingen\\
&Feministvikingen\\
17. Den bleka&Uti min mage\\
\hspace{17pt}dödens dryck&Då verka lätt
\end{tabularx}
\end{table}

\vspace{-5pt}

\vspace{-10pt}
\begin{table}[!ht]
\begin{tabularx}{1\textwidth}{l X}
\Large $\varepsilon$ (epsilon) - Visor till vin&\\
\end{tabularx}
\end{table}
\begin{table}[!ht]
\begin{tabularx}{1\textwidth}{l X}
1.&Vinets lov\\
2.&Feta fransyskor\\
3.&Vinvisa (Har ni sett på attan)\\
4.&Vinet skänker\\
5.&Du gamla vin\\
6.&Elysisk längtan\\
7.&Bordeaux, Bordeaux\\
8.&Spegelvisa\\
9.&Röd vitamin\\
10.&Portvinsvisa\\
11.&Röda vinet\\
&Vinet väntar
\end{tabularx}
\end{table}
\begin{table}[!ht]
\begin{tabularx}{1\textwidth}{l X}
\Large $\zeta$ (zeta) - Visor till punsch&
\end{tabularx}
\end{table}
\begin{table}[!ht]
\begin{tabularx}{1\textwidth}{l X}
1.&Punschen kommer\\
2.&Punschkanon\\
3.&Punschschottis\\
4.&Punschens lov\\
5.&Jag gillar punschen\\
6.&Imperial punsch
\end{tabularx}
\end{table}
\begin{table}[!ht]
\begin{tabularx}{1\textwidth}{l X}
7.&Djungelpunsch\\
8.&Vi vill ha punsch\\
9.&Punsch, punsch\\
10.&Studiemedelsrondo\\
11.&FESTU:s punschvisa\\
12.&Visa en torsdagskväll\\
13.&Sveriges Arraktionalhymn\\
1/$\varepsilon$.&Punschfinalen\\
$\infty$.&Sista punschvisan
\end{tabularx}
\end{table}
\begin{table}[!ht]
\begin{tabularx}{1\textwidth}{l X}
\Large $\eta$ (eta) - Visor till andra drycker&
\end{tabularx}
\end{table}
\begin{table}[!ht]
\begin{tabularx}{1\textwidth}{l X}
1.&Skål för vattnet\\
2.&En kan dricka vatten\\
3.&Nu tar vi rom\\
4.&Däj-o\\
5.&Mjölk\\
6.&Mjölksång\\
7.&Hyllningsvisa till absinten\\
8.&Schottis på Valhall\\
9.&Häflåten\\
10.&Kaffe\\
11.&Whiskyn\\
12.&1, 2, 3, Whisky!\\
13.&Kahlua
\end{tabularx}
\end{table}
\begin{table}[!ht]
\begin{tabularx}{1\textwidth}{l X}
\Large $\theta$ (theta) - Blöta visor&\\
\normalsize Salomos ordspråk&
\end{tabularx}
\end{table}
\begin{table}[!ht]
\begin{tabularx}{1\textwidth}{l X}
1.&Störthärligt full\\
2.&Jag var full en gång\\
3.&Bär ner mig till sjön
\end{tabularx}
\end{table}

\end{document}
